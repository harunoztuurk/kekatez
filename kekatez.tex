
\textbf{Önerme 3.1} \\
$F_1 = (\sin c, 0, \cos c)$, $F_2 = (-\sin c, 0, \cos c)$, $S^2$ üzerinde iki nokta olsun. Küresel elips, $|\overset{\frown}{F_1P}| + |\overset{\frown}{F_2P}| = 2a$ koşulunu sağlayan $S^2$ üzerindeki $P$ noktaları tarafından belirlenir. Buradan bu küresel elips, $S^2$ ile
\[
\frac{x^2}{\tan^2 a} + \frac{y^2}{\tan^2 b} = z^2
\]
denklemiyle verilen eliptik koninin arakesitidir. Burada $0 < c < \frac{\pi}{2}$ ve $\cos a = \cos b \cos c$ dir   .
\textbf{Uyarı 3.1} \\
$F_1 = (\sin c, 0, \cos c)$ ve $F_2 = (-\sin c, 0, \cos c)$, küresel elipsin odak noktaları olsun. $0 < b < a$ için:
\begin{itemize}
    \item $F'_1 = (\sin a, 0, \cos a)$ noktası, küresel elipsin \textbf{asal ekseninin uç noktasıdır}.
    \item $F'_2 = (0, \sin b, \cos b)$ noktası, küresel elipsin \textbf{yedek ekseninin uç noktasıdır}.
\end{itemize}
  .

\textbf{Sonuç 3.2} \\
$S^2$ üzerindeki herhangi bir $C$ küresel koniği için $S^2$ üzerinde $F_1$, $F_2$ noktaları ve $a < \frac{\pi}{2}$ pozitif bir sayı vardır öyle ki:
\[
C = \{P \mid |\overset{\frown}{PF_1}| + |\overset{\frown}{PF_2}| = 2a\} \cup \{P \mid |\overset{\frown}{PF^*_1}| + |\overset{\frown}{PF^*_2}| = 2a\},
\]
\[
C = \{P \mid |\overset{\frown}{PF^*_2}| - |\overset{\frown}{PF_1}| = \pi - 2a\} \cup \{P \mid |\overset{\frown}{PF_1}| - |\overset{\frown}{PF^*_2}| = \pi - 2a\},
\]
burada $F^*_1$ ve $F^*_2$, sırasıyla, $F_1$ ve $F_2$'nin antipodal noktalarıdır   .

\textbf{Şekil 3.4:} 
\begin{figure}
    \centering
    \includegraphics[width=0.5\linewidth]{Ekran Resmi 2025-05-05 ÖS 6.53.33.png}
\end{figure}


$a = \frac{\pi}{3}$, $b = \frac{\pi}{4}$ ve $c = \frac{\pi}{4}$ değerleri için, küresel elipsin asal (a) ve yedek (b) ekseni.

2012'de Kopacz   , Maeda   'ya ek olarak, küresel parabol için aşağıdaki eşitliği vermiştir:
\[
\big||\overset{\frown}{PF_1}| - |\overset{\frown}{PF^*_2}|\big| = \frac{\pi}{2} = \text{sabit}.
\]

2012'de Altunkaya, Yaylı, Hacısalihoğlu ve Arslan   , sırasıyla, aşağıda verilen tek parametreli küresel elips, hiperbol ve parabol denklemlerini elde etmiş ve bununla ilgili örnekler vermiştir:
\[
\vec{F}_1, \quad \vec{F}_2, \quad \vec{F}_3 = \frac{\vec{F}_1 \times \vec{F}_2}{\|\vec{F}_1 \times \vec{F}_2\|}
\]
lineer bağımsız vektörler ve $\theta$, konikler üzerindeki keyfi bir $P$ noktası arasındaki açı olmak üzere:
\[
X(t) = \frac{\cos t - \cos 2c \cos(2b - t)}{\sin^2 2c} \vec{F}_1 + \frac{-\cos 2c \cos t + \cos(2b - t)}{\sin^2 2c} \vec{F}_2 + \cos \theta \vec{F}_3,
\]
\[
Y(t) = \frac{\cos t - \cos 2c \cos \phi}{\sin^2 2c} \vec{F}_1 + \frac{-\cos 2c \cos t + \cos \phi}{\sin^2 2c} \vec{F}_2 + \cos \theta \vec{F}_3,
\]
\[
Z(t) = \frac{\cos t + \cos 2c \sin t}{\sin^2 2c} \vec{F}_1 + \frac{-\cos 2c \cos t - \sin t}{\sin^2 2c} \vec{F}_2 + \cos \theta \vec{F}_3,
\]
burada $|t - \phi| = 2b$ ve $b$, $c$ sabittir   .

\section*{Kaynakça}

1. Altunkaya, B. *Küresel Konikler ve Uygulamaları*, Doktora Tezi, Ankara Üniversitesi Fen Bilimleri Enstitüsü, 2012.

2. Altunkaya, B.; Yaylı, Y.; Hacısalihoglu, H. H.; Arslan, F. *Equations of the spherical conics*, Electronic Journal of Mathematics and Technology, 2011, 5, no.3, 330-341.

3. Altunkaya, B.; Yaylı, Y.; Hacısalihoglu, H. H.; Arslan, F. *One-Parameter Equations of Spherical Conics and Its Application*, Journal of Mathematics Research, 2014, 6, no.4, 77-84.

4. Balcı, M. *Analitik Geometri*, Balcı Yayınları, Ankara, 2007.

5. Dirnbock, H. *Absolute polarity on the sphere; conics; loxodrome; tractrix*, Mathematical Communication, 1999, 4, 225-240.

6. Izumiya, S.; Takeuchi, N. *New Special Curves and Developable Surfaces*, Turk J. Math., 2004, 28, 153-163.

7. Izumiya, S.; Takeuchi, N. *Generic properties of helices and Bertrand curves*, Journal of Geometry, 2002, 74, 97-109.

8. Hacısalihoglu, H. H. *Diferensiyel Geometri 1. Cilt*, Fen Fakültesi, Beşevler-Ankara, 2000.

9. Hacısalihoglu, H. H. *Diferensiyel Geometri 2. Cilt*, Fen Fakültesi, Beşevler-Ankara, 2000.

10. Karger, A.; Novak, J. *Space Kinematics and Lie Groups*, Gordon and Breach Science Publishers, 1985.

11. Gibson, C. G. *Elementary Geometry of Algebraic Curves, an Undergraduate Introduction*, Cambridge University Press, Cambridge, 2001.

12. Hazewinkel, M. *Antipodes*, Encyclopedia of Mathematics, Springer, London, 2001.

13. Maeda, Y. *Spherical conics and the fourth parameter*, KMITL Sci. J., 2005, 5, no.1, 165-171.

14. Maeda, Y. *How to project spherical conics into the plane*, 2006.

15. Namikawa, Y. *Spherical surfaces and hyperbolas*, Sugaku, The Mathematical Society of Japan, 1959/60, 11, no.1, 22-24.

16. Kopacz, P. *On geometric properties of spherical conics and generalization of Pi in navigation and mapping*, Geodesy and Cartography, 2012, 38, no.4, 141-151.

17. Sabuncuoğlu, A. *Analitik Geometri*, Nobel Yayınları, Ankara, 2009.

18. Sabuncuoğlu, A. *Diferensiyel Geometri*, Nobel Yayınları, Ankara, 2006.

19. Sykes, G. S.; Peirce, B. *Spherical Conics*, Proceedings of the American Academy of Arts and Sciences, 1878, 13, 375-395.

20. Xiong, J. *Geometry and singularities of spatial and spherical curves*, PhD Thesis, Hawaii University, 2004.

21. Wong, Y. C. *On an explicit characterization of spherical curves*, Proceedings of the American Mathematical Society, 1972, 34, no.1, 239-242.

\end{document}
